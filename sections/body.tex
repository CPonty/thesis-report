\mainmatter

% =====================================================================

\chapter{Introduction}

A block diagram is a model of a system in which components or functions are represented by blocks, and relationships or connections between them by connecting lines. Block diagram models of engineering systems are heavily used by designers and educators in many fields, including hardware design, electronics design, software engineering, control systems and others.
\\

Sketching block diagrams (or block diagram equivalent designs) on a static surface (such as paper or a whiteboard) is a common part of the design process for engineering students, educators and professionals. Often the system being designed is subsequently analysed and simulated using a computer aided tool, requiring the sketched design to be manually reconstructed on a computer. It should be possible to reduce or eliminate this delay between sketch and simulation and thus increase productivity for system designers using an automated, computer aided process.
\\

Today there are plenty of available systems capable of handwriting recognition for various engineering designs. However, these are generally specialised for taking input from a stylus or touchscreen, and/or recognition of symbols and designs for a specific engineering domain.
\\

\textit{BlocSim}\footnote{BlocSim - http://www.github.com/CPonty/BlocSim} is a proof-of-concept for a sketch-to-simulation solution capable of servicing a broad range of engineering domains. Its role is to visually capture a sketch of a block-diagram-like model and update external simulation tools with an adapted model in real-time. For the sake of simplicity, capturing textual annotations adjacent to blocks is outside its scope.
\\

An open-source prototype of BlocSim shall be built which achieves this functionality for simple whiteboard sketches. BlocSim will be built with extensibility in mind, as expanding its ability to interface with third-party software will require updates over time.


\begin{comment}

\end{comment}

% =====================================================================

\chapter{Background}

\section{Block Diagrams in Engineering}

%\subsection{Block Diagram Equivalent Drawings}
%TODO



Similar arguments can be made for digital logic gate schematics and many other domains where blocks are substituted for symbols. Basic schematic and other technical drawings specific to many engineering domains can be modelled by a generic block diagram.

\begin{figure}[ht!]
\includegraphics[width=70mm]{images/blockDiagram.png}
\hspace{0.5 cm}
\includegraphics[width=50mm]{images/circuitDiagram2.png} \\
\includegraphics[width=50mm]{images/controlSystem.png} 
\hspace{0.5 cm}
\includegraphics[width=50mm]{images/fbd.jpg} 
\centering
\caption{Block diagram (top left) \cite{wiringDiagrams}, circuit diagram (top right) \cite{engineeronadisk}, control system block diagram (bottom left) \cite{engineeronadisk}, function block diagram (bottom right) \cite{wikicommonsFbd}}
\label{im:blockExample}
\end{figure}

%\clearpage

\subsection{Simulation}

\subsubsection{Simulink}

Mathworks' \textit{Simulink toolbox} \cite{simulink} is a powerful tool for block diagram system simulation. It provides a block diagram environment for simulation and design of system models across a variety of domains. It includes a graphical editor, customisable library of blocks, solver tools for modelling and simulation of systems, and the ability to incorporate Matlab algorithms in the model. Simulink is an ideal candidate for integration with BlocSim.

\begin{figure}[ht!]
\includegraphics[width=75mm]{images/simulink.png} 
\centering
\caption{Simulink block diagram model}
\label{im:simulink}
\end{figure}

\subsubsection{LTSpice}

LTSpice \cite{ltspice}, a SPICE circuit schematic capture and simulation tool built by Linear Systems, is a typical example of a simulation tool which could take input from a block diagram model. Looking at the simple circuit example in Figure \ref{im:ltspice}, we can see that the schematic consists of components, lines and textual annotations on components. The SPICE netlist\footnote{Netlist - In the context of electronics design, a netlist conveys information about the instances (parts) used, how their pins/ports are connected, and some other attributes.} file representing the schematic is a straightforward sequence of human readable components, wire routes and component properties. Aside from component properties, a general purpose block diagram model including connector geometry contains sufficient information to be translated into an LTSpice circuit schematic, with block component type IDs map to circuit component types.

Similar arguments can be made for other tools and other engineering domains.

%\footnote{Netlist - In the context of electronics design, a netlist conveys information about the instances (parts) used, how their pins/ports are connected, and some other attributes.}

%\subsection{LTSpice}
%LTSpice, and spice 
%\cite{ltspice}

\begin{figure}[ht!]
\centering
\includegraphics[width=50mm]{images/ltspice1.png}
\includegraphics[width=50mm]{images/ltspice2.png}
\caption{Simple LTSpice circuit with corresponding netlist file \cite{ltspice}}
\label{im:ltspice}
\end{figure}

\clearpage

\section{Block Diagram Sketch Capture}

%TODO

\subsection{Engineering Sketch Recognition \& Simulation}

%TODO

\begin{figure}[ht!]
\centering
\includegraphics[width=75mm]{images/sketchInterpSoftware.jpg}
\caption{Recognition of stylus-drawn electrical circuit components \cite{sketchInterpSoftware}}
\label{im:sketchInterpSoftware}
\end{figure}

\clearpage

\subsubsection{QuickDiagram}

%TODO

\subsection{Electronic Whiteboards}

%TODO

\subsection{Block Component Identification}

%TODO

%introduce problem fiduciary markers
%markers
	%artag
%ocr
	%don't want to use OCR for many cases - math symbols, domain specific symbols
%barcodes (1D)
%QR codes (2D)

\subsubsection{Fiduciary Markers}

%TODO

\begin{figure}[ht!]
\centering
\includegraphics[width=50mm]{images/artoolkit1.png}
\includegraphics[width=50mm]{images/artoolkit2.jpg}
\caption{Custom fiduciary markers for ARToolKit \cite{fiducialPic}}
\label{im:fiducialPic}
\end{figure}

\clearpage

\subsubsection{OCR}

%TODO

\subsubsection{Barcodes}

%TODO

%\cite{sketchInterpSoftware}
%\cite{quickDiagram}
%\cite{quickDiagramDemo}
%\cite{interactiveWhiteboards}
%\cite{interactiveWhiteboards2}
%\cite{panasonic}
%\cite{ledLcd}
%\cite{handDrawnCircuits}
%\cite{artag}
%\cite{bestFiducial}
%\cite{artoolkit}
%\cite{ltspice}

%\cite{fiducial}
%\cite{fiducialAR}
% G. Schall et al
%\cite{colorFiducial}
% H. Bagherinia et al

%\begin{figure}[ht!]
%\centering
%\includegraphics[width=50mm]{images/fiducialPic.jpg}
%\caption{Example of fiducials present on a \gls{pcb} \cite{fiducialPic}}
%\label{im:fiducialPic}
%\end{figure}


% =====================================================================

\subsubsection{Frame Markers}
\label{ch:back:borders}

As an alternative to symbol, text or traditional barcode recognition, a special type of fiducial can be constructed called a frame marker. The concept is that the unique ID of the marker is encoded in a pattern along its border, rather than in the central contents area. This means that the majority of the marker surface is not required during image recongnition, allowing a user to place any image, text, symbolic or other content here as they see fit without affecting the recognition of the marker.

An extremely simple version of this concept could involve placing a small dot in the corner of the marker which corresponds to its unique ID. This gives 4 possible marker types. A more comprehensive alternative could encode the unique ID in a continuous, barcode-like pattern along the inner border of the marker. This could allow for a range of hundreds of unique IDs, only limited by the resolution of the marker in the image.

Vuforia\footnote{Vuforia - \url{https://www.vuforia.com}}, a computer vision platform for augmented reality (\gls{ar}) on mobile devices, implements frame marker recognition. For the marker type in Figure \ref{im:frameMarker}, a range of 512 unique IDs are available, with relatively low processing power required for recognition \cite{vuforia}. However, verbatim use of this implementation does not suit BlocSim - Vuforia's markers require a clearance area around the frame border free of graphical features, while in a block diagram sketch the connector lines will violate this space.

Overall, this is a highly promising method for encoding block identities.
%, allowing a very large range of types and increasing reliability by removing the need for colour detection.

\begin{figure}[ht!]
\centering
\includegraphics[width=50mm]{images/frameMarker.png}
\caption{Vuforia Frame Marker fiducial example \cite{vuforia}}
\label{im:frameMarker}
\end{figure}

\clearpage

%\cite{vuforia}

% =====================================================================

\section{Open Source Software}

A key benefit of writing free and open source software (\gls{foss}) comes from creating opportunity for others to contribute to a project. However, this requires the developers to create an environment suitable for collaboration, which involves more than simply writing nicely formatted \& documented code.
\\

\subsection{Contributors}

Contributions to an active open-source project come from many contributor roles. Aside from \textit{developers} writing and documenting code, there should be \textit{testers} actively using and testing the project, people providing \textit{user support} by answering questions (e.g. on a forum), \textit{bug reporters} and regular \textit{users}. Any one contributor can fill several roles. A final and important role are the project's \textit{observers}, who may be considering becoming contributors.

Clearly all of these roles are important ones. Therefore, it is beneficial to setup infrastructure to support all of them. This can include, but is not limited to:
\begin{description}
	\item[Ease of Access] - providing a convenient means for users to acquire \& run the software
	\item[Project Monitoring] - keeping all roles informed with the status of the project, such as the roles of developers, changesets, project direction
	\item[User Support] - providing a means for users to learn about the project, seek help with issues and contact developers
	\item[Issue Tracking] - providing a means for users and developers to create and monitor bug reports or suggestions for new features. This functionality is at the core of open source development
	\item[Centralised source control] - ensure developers have a central 'source of truth', with the ability to effectively manage distributed development and merge work
\end{description}

Taking steps to support contributors is a key to success for all open source projects. \cite{openSource2}. 

\textit{GitHub}\footnote{GitHub - \url{https://www.github.com/}} is a highly valuable and widely used online space for open source software projects. It provides a \textit{Git}\footnote{Git - \url{http://www.git-scm.com/}} version control system, facilitating branches of development and the ability to rollback/merge changes, a wiki page for documentation, release management and an issue tracker.

\vspace{0.5 cm}

\subsection{Software Evaluation \& Selection}

Selection of existing software, libraries and tools for use in a project must be carefully considered. Often there exists a wide range of possible off-the-shelf components which can fill a given requirement, and each must be evaluated on its merits to the project. Evaluation can involve, but is not limited to, researching views and opinions from the open source community and reading source code and documentation. Selection criteria are broad, and may include source code quality, documentation quality and completeness, maturity, stability, cross-platform support, ease of use \& maintenance, and importantly the permissiveness of its licensing. \cite{openSource}
\\

Given that BlocSim's key functionality requires converting block models to be compatible with a variety of third party software, it will require maintenance and ongoing work over time. A FOSS model will serve the project well.

% =====================================================================


\section{Computer Vision Software}

As BlocSim's core functionality involves interpreting hand-drawn sketches, it will require the ability to perform image processing on a captured sketch.

\subsection{Computer Vision Frameworks}

A large number of free, off-the-shelf image processing and computer vision libraries are available to this project. In order to minimise time required to implement the BlocSim prototype, evaulation of options focuses on frameworks I have previous experience with. Two such frameworks were evaluated for this project: Mathworks' \textit{Computer Vision System toolbox} \cite{matlab} and the OpenCV Foundation's \textit{Open Source Computer Vision Library} (OpenCV) \cite{opencv}.
\\

In terms of performance, OpenCV has a strong advantage over most freely available frameworks.
Its stated purpose is to "provide a common infrastructure for computer vision applications and to accelerate the use of machine perception" \cite{opencv}. OpenCV is a proven product, with an active community of over 47,000 users and extensive use among well-established software companies. OpenCV began its life as a project in Intel's labs, resulting in Intel's Integrated Performance Primitives (IPP), which include assembly routines for computer vision optimised by hand \cite{opencvBook}. While OpenCV APIs are available for several languages including Java and python, the core functionality is written in compiled, optimised C++. For these reasons, OpenCV generally outperforms other freely available open source libraries \cite{opencvBench}.
\\

Working in Matlab offers different advantages. Matlab's Computer Vision System toolbox actually integrates the OpenCV library into Matlab, allowing access to Matlab's convenient visualisation and data processing capabilities, higher levels of abstraction and an extended image processing library \cite{matlabCV}. However, working with Matlab would present several problems:

\begin{itemize}
\item Matlab is commercial software with considerable purchase cost, which would erode the project's open-source value
\item As a large software package running in a Java virtual machine, Matlab's performance does not match that of utilising OpenCV directly
\item Writing features of BlocSim outside of computer vision will be more convenient in a standalone program than executing in a data analysis tool,
\end{itemize}

\begin{figure}[ht!]
\centering
\includegraphics[width=25mm]{images/OpenCVlogo.png}
\caption{The OpenCV Logo \cite{opencv}}
\label{im:cvlogo}
\end{figure}

%\subsection{Techniques}


%select OpenCV; discuss useful features it has with footnotes
% colour thresholding
% contour heirarchy
% adaptive thresholding

%\cite{matlab}
%\cite{opencvBench}
%%\cite{opencvBook}



% =====================================================================

\chapter{BlocSim Specification}
\label{ch:spec}



\section{Solution Concept}

\begin{figure}[ht!]
\centering
\includegraphics[width=150mm]{images/BlocsimBD1.jpg}
\caption{BlocSim System Design \cite{blocsimPoster}}
\label{im:BlocsimBD1}
\end{figure}

\clearpage



\section{Block Diagram Capture}



\subsection{Fiducial Design}


%TODO


\begin{figure}[ht!]
\centering
\includegraphics[width=100mm]{images/BlocsimBloc1.jpg}
\caption{BlocSim Block Diagram format (sketch)}
\label{im:BlocsimBD1}
\end{figure}

\begin{figure}[ht!]
\centering
\includegraphics[width=100mm]{images/BlocsimBloc2.jpg}
\caption{BlocSim Block Diagram format (model)}
\label{im:BlocsimBD2}
\end{figure}

\begin{figure}[ht!]
\centering
\includegraphics[width=100mm]{images/BlocsimBloc3.jpg}
\caption{BlocSim Block Diagram format (block types A,B,C,D mapped to Digital Logic gates)}
\label{im:BlocsimBD3}
\end{figure}

\clearpage



\subsection{Computer Vision Requirements}
\label{sec:techSelection}


%TODO


%\newpage

\section{Technology Selection}

\begin{table}[ht!]
	\center
	\begin{tabular}{$r^l} %$
		\hline
		\rowstyle{\bfseries}
		Component & Selected Technology \\
		\hline \\
		Primary Programming Language & \textbf{Python} \footnotemark[1] \\
		Computer Vision Package & \textbf{OpenCV} \footnotemark[2] \\
		Human Interface & Web control panel \\
		Web Framework & \textbf{Tornado} \footnotemark[3] \\
		Communication/Storage Format & \textbf{\gls{json}} \footnotemark[4] \\
		Remote Access & \textbf{JSON-RPC} \footnotemark[5] protocol via \textbf{TornadoRPC} \footnotemark[6] \\
		Client-Server Communication & \textbf{Websockets} \footnotemark[7], using \textbf{SockJS} \footnotemark[8] \& \textbf{SockJS-Tornado} \footnotemark[9] \\
		Block Diagram Publishing & \textbf{Mosquitto} \footnotemark[10] Publish/Subscribe Broker \\
		\\
		\hline
	\end{tabular}
	\caption{Selection of technologies for the BlocSim prototype}
	\label{tab:techSelect}
\end{table} 



%TODO



%\cite{facebookTornado}
%\cite{socketioBugs}
%\footnotemark[11]
%\footnotemark[12]
%\footnotemark[13]
%\footnotemark[14]
\footnotetext[1]{Python - \url{https://www.python.org/}}
\footnotetext[2]{OpenCV - \url{http://opencv.org/}} 
\footnotetext[3]{Tornado - \url{http://www.tornadoweb.org/en/stable/}}
\footnotetext[4]{JSON - \url{http://json.org/}}
\footnotetext[5]{JSON-RPC - \url{http://json-rpc.org/wiki/specification}}
\footnotetext[6]{TornadoRPC - \url{https://github.com/joshmarshall/tornadorpc}}
\footnotetext[7]{Websockets - \url{http://www.websocket.org/}}
\footnotetext[8]{SockJS - \url{https://github.com/sockjs/sockjs-client}}
\footnotetext[9]{SockJS-Tornado - \url{https://github.com/mrjoes/sockjs-tornado}}
\footnotetext[10]{Mosquitto - \url{http://mosquitto.org/}}
\footnotetext[11]{Flask - \url{http://flask.pocoo.org/}}
\footnotetext[12]{Socket.io - \url{http://socket.io/}}
\footnotetext[13]{Redis - \url{http://redis.io/}}
\footnotetext[14]{Meteor - \url{https://www.meteor.com/}}

\clearpage

% =====================================================================

\section{Open-Source Delivery}

The BlocSim source, release build, issue tracking and documentation are to be publicly hosted online at GitHub \cite{blocsim}\cite{github}. Having a centralised, publicly visible location for these services and material provides both exposure for the project and valuable support infrastructure for any future work on the project (by its original author or others) which may take place. 

Additionally, every effort has been made to select only free and open source libraries and dependencies, making the project more accessible to developers.

BlocSim is to be licensed under GPLv2 \cite{gplv2}, a permissive license which guarantees freedom to share and change the software, appropriate for encouraging and facilitating open source development.
\\


% =====================================================================

\chapter{Implementation}

Implementation of the BlocSim prototype followed the choices of technologies and libraries laid out in Section \ref{sec:techSelection}. The primary focus was on rapid development and maximising user-friendliness of the \gls{ui} design. The result is a monolithic Python program covering all functionality, with a web control panel powered and styled by JQuery\footnote{JQuery - \url{http://jquery.com} and \url{http://jqueryui.com}}.
\\

To start BlocSim, users simply navigate to the \texttt{demo} folder and execute \texttt{./blocsim.py}. The webserver and associated services will start automatically. BlocSim will connect to the first available webcam, load its stored configuration from a JSON file (\texttt{config/config.db}), and start processing webcam frames.
\\

\newpage

Users can access the control panel and view the video stream from any computer in the network: simply navigate to \texttt{http://<host address>:8080}.

\begin{figure}[ht!]
\centering
\includegraphics[width=50mm]{images/BlocsimLogo.png}
\caption{The BlocSim Logo \cite{blocsim}}
\label{im:blocsimlogo}
\end{figure}

\vspace{0.5 cm}

A high-resolution (1920x1080 pixels), consumer-affordable webcam (Logitech C920 HD\footnote{Logitech C920 HD Webcam - \url{http://www.logitech.com/en-au/product/hd-pro-webcam-c920}}) was acquired to ensure capture resolution was not the limiting factor in the computer vision component. The camera is approximately 1m from the board.

\begin{figure}[ht!]
\centering
\includegraphics[width=175mm]{images/photo2.jpg}
\caption{Camera and whiteboard demonstration setup}
\label{im:hardware1}
\end{figure}

\clearpage

A demonstration block diagram has been constructed on a magnetic whiteboard. Aside from a loop of four valid electrical-circuit-themed blocks, the board includes a number of test features:
\begin{itemize}
\item a connector line with no connections (right)
\item a circular red block (right)
\item a 'block within a block' (left of circle)
\item a 'loose' block component (no connections, top)
\item stray image features in the same colour as block diagram components, but invalid configuration ('Bloc Sim' title, center)
\end{itemize}

The 'V1' and 'R1' components are magnetic cards, while all other features on the whiteboard consist of whiteboard marker directly drawn onto the board. There is little practical difference between the two, as the magnetic cards have similar visual properties (white, reflective, rendered with whiteboard marker)

\begin{figure}[ht!]
\centering
\includegraphics[width=75mm]{images/photo3.jpg}
\includegraphics[width=85mm]{images/photo4.jpg}
\caption{Logitech C920HD 1080p Camera; close-up of block diagram on whiteboard}
\label{im:hardware2}
\end{figure}

\clearpage

\section{Features \& Usage}

The BlocSim web control panel can be seen in Figure \label{im:screenshot_cv1}. Three tabs (Webcam, Publish/Subscribe, RPC) are available (top). Basic controls are available on a sidebar to the left, including pausing the video feed, switching webcams, video feed quality and selectively enabling publish/susbscribe streaming. The sidebar also signals if a feature on the server is unavailable (green/orange/red lights).

In the webcam tab, a select menu (combo box) above the video feed allows the user to select one of 16 streams to view. These images are generated during each stage of analysis of the webcam sketch, enabling users to better appreciate how BlocSim works and, in the case of sketch capture failing, troubleshoot their capture configuration.
\\

\begin{figure}[ht!]
\centering
\includegraphics[width=175mm]{images/screenshot_cv1.png}
\caption{The BlocSim Web Control Panel \cite{blocsim}}
\label{im:screenshot_cv1}
\end{figure}

\newpage

Scrolling down, a calibration panel provides sliders which update the configuration of the computer vision processing in real time. This is primarily for developer debugging purposes, but also allows users to easily adjust settings if the lighting changes significantly, and crop the video feed to only show the block diagram on a whiteboard. A row of buttons provides additional functionality, allowing the user to load and save this configuration, or save the current block diagram model (including all 16 processed images) to a folder.
\\

At the bottom of the Webcam tab, the last JSON message received from the server is displayed, showing details such as the framerate, frame dimensions and frame filesize.
\\

\begin{figure}[ht!]
\centering
\includegraphics[width=170mm]{images/screenshot_calib2.png} %\\
%\vspace{0.1 cm}
\includegraphics[width=170mm]{images/screenshot_info2.png}
\caption{BlocSim Control Panel - calibration sliders and server message \cite{blocsim}}
\label{im:screenshot_calib}
\end{figure}

\newpage
The Publish/Subscribe tab shows the JSON model corresponding to the block diagram in the current frame. This model (minus the space formatting) is published to the Mosquitto broker on every frame.
\\

To subscribe to the model, open a command line window and run: \\ \texttt{mosquitto\_sub -t blocsim}

\begin{figure}[ht!]
\centering
\includegraphics[width=150mm]{images/screenshot_pubsub.png}
\caption{BlocSim Control Panel - Block Diagram model \cite{blocsim}}
\label{im:screenshot_pubsub}
\end{figure}

\newpage

The final tab, RPC, is simply a log of all remote procedure calls made to the server. In order to demonstrate this feature, some of the UI buttons are implemented as remote procedure calls. An example python JSON-RPC script is show on-screen; a full list of available procedures is available on GitHub.

\begin{figure}[ht!]
\centering
\includegraphics[width=150mm]{images/screenshot_rpc.png}
\caption{BlocSim Control Panel - Remote Procedure Call server status \cite{blocsim}}
\label{im:screenshot_rpc}
\end{figure}

%\clearpage

\newpage

\subsection{Computer Vision}

The following pages are an abridged walkthrough of the computer vision processes employed by BlocSim. Not all image frames are reviewed.

\begin{figure}[ht!]
\centering
\includegraphics[width=140mm]{images/frame1.jpg}
\caption{BlocSim output image (1) - cropped frame \cite{blocsim}}
\label{im:frame1}
\end{figure}

\newpage
Image 3: shows the cropped frame after applying threshold bounds for hue, saturation and brightness (value) in the HSV colour range \& removing noise

\begin{figure}[ht!]
\centering
\includegraphics[width=140mm]{images/frame3.jpg}
\caption{BlocSim output image (3) - red \cite{blocsim}}
\label{im:frame3}
\end{figure}

\newpage
Image 4: shows the set of rectangular red contours which classified as potential block markers. Note that the circle has been rejected for being non-rectangular. The small hidden red shape on the right was rejected as it is not a closed, hollow shape. The bounding regions of the blocks are minimum rectangles which follow their axis of rotation. 

Also note that the whiteboard is not directly facing the camera (it is out-of-plane). This small distortion effect does not impact the recognition of the block diagram model.

\begin{figure}[ht!]
\centering
\includegraphics[width=140mm]{images/frame4.jpg}
\caption{BlocSim output image (4) - red block \cite{blocsim}}
\label{im:frame4}
\end{figure}

\newpage
Image 8: A similar procedure to images 2..4 has been applied to Green colour regions already. Only solid, approximately round shapes inside a red rectangle from (4) are accepted.

The green dots identify the block types. Note the red block in the centre has been rejected, as its only green dot is not near the corner of the shape. The red contour surrounding the word 'Sim' is also being ignored, as it did not contain an identifying dot.

\begin{figure}[ht!]
\centering
\includegraphics[width=140mm]{images/frame8.jpg}
\caption{BlocSim output image (8) - block ID \cite{blocsim}}
\label{im:frame8}
\end{figure}

\newpage
Image 12: Thresholding has been applied to black regions of the image. The bounds on what is considered a 'black' region are very broad to account for reflections on the whiteboard. 

The black regions are candidates for connector lines. They are dilated (expanded), and any part of the dilated regions intersecting a red block is considered a node. Any connectors without nodes will be ignored later.

\begin{figure}[ht!]
\centering
\includegraphics[width=140mm]{images/frame12.jpg}
\caption{BlocSim output image (12) - connector lines (mask) \cite{blocsim}}
\label{im:frame12}
\end{figure}

\newpage
Image 13: Nodes have been marked on the image.

\begin{figure}[ht!]
\centering
\includegraphics[width=140mm]{images/frame13.jpg}
\caption{BlocSim output image (13) - connection nodes \cite{blocsim}}
\label{im:frame13}
\end{figure}

\newpage
Image 15: All information so far has been combined, and paths between nodes and blocks drawn. For each connector line candidate from (12), all nodes within its perimeter are joined together. 

The result is a set of 4 identified blocks joined by a circular path, with a 5th block disconnected (top).

\begin{figure}[ht!]
\centering
\includegraphics[width=140mm]{images/frame15.jpg}
\caption{BlocSim output image (15) - block diagram overlay \cite{blocsim}}
\label{im:frame15}
\end{figure}

\newpage
Image 16: This final image shows the block diagram model without the background image.

\begin{figure}[ht!]
\centering
\includegraphics[width=140mm]{images/frame16.jpg}
\caption{BlocSim output image (16) - block model \cite{blocsim}}
\label{im:frame16}
\end{figure}

\clearpage
\newpage

Finally, we have the JSON Block diagram model corresponding to the previous set of images. Only a sample of the blocks and nodes are shown for brevity. The model consists of a list of blocks and list of nodes, each of which is assigned an ID, position, and list of connecting components. Block markers also describe their shape/size and block type.

\begin{figure}[ht!]
	\fontsize{8pt}{8pt}
	\singlespacing
	\begin{mdframed}
		\verbinput{images/example.json}
	\end{mdframed}
	\caption{Sample of JSON Block Diagram Model output (not all blocks, nodes shown)}
	\label{tab:json}
\end{figure}

\clearpage




% =====================================================================

\section{Technical Architecture}

Figure \ref{im:BlocsimBD3} shows the overall architecture of the BlocSim prototype's implementation. Broadcast messages are sent over the websocket connection by both the client (browser) and server at 10Hz. On each message, the server updates the browser client with all relevant information related to the last-processed frame. The client browser sends the server the current state of all its UI elements, including computer vision configuration sliders.
\\

The server stores all configuration items in a json keystore. This keystore is updated every time a client's broadcast packet arrives. While this is an acceptable implementation for a single client, this model will have to be updated to support multiple clients.

\begin{figure}[ht!]
\centering
\includegraphics[width=170mm]{images/BlocsimBD3.jpg}
\caption{BlocSim's Architecture}
\label{im:BlocsimBD3}
\end{figure}

Initially, BlocSim was built to stream the webcam feed in an .MJPEG format. However, MJPEG connections proved difficult to manage, partially due to a bug in the Webkit implementation for some browsers where the only way to disconnect the .MJPEG stream from the server is to close the tab.
\\

Instead, each websocket connection has associated with it a currently streaming image ID. Within each broadcast message, this image is encoded into a base64 string (which increases the file size by x1.33). This image is assigned to the 'data' URI of a .jpg image by javascript upon arrival at the browser.
\\

This is an acceptable implementation for the low frame rates required watching a primarily static diagram.

\begin{figure}[ht!]
\centering
\includegraphics[width=170mm]{images/BlocsimBD4.jpg}
\caption{Process flow for video stream}
\label{im:BlocsimBD4}
\end{figure}

%\newpage
%\clearpage

\noindent The Python webserver's functionality is split into several threads, including:

\begin{description}
	\item[Main Thread] - Executes the Tornado webserver IO Loop. Includes handling websockets and Remote Procedure Calls
	\item[Publish Thread] - Executes the Mosquitto client IO Loop
	\item[Capture Thread] - Reads from the camera when signalled
	\item[Webcam Thread] - Handles webcam connection/disconnection and storing frames
	\item[Processing Thread] - Uses OpenCV to generate images and the block model
	\item[Timer Thread] - Limits the webcam capture frame rate
\end{description}

\noindent These threads communicate via a series of event signallers and shared resources, such as the keystore database and image processing output.

\vspace{1 cm}

\noindent For more information on BlocSim's implementation and usage, see the resources outlined in Appendix \ref{ch:appendix}.


% =====================================================================

\chapter{Project Evaluation}

\section{Completion}

The creation of the BlocSim prototype according to speicifications laid out in Chapter \ref{ch:spec} was generally completed successfully. The system is very user friendly, very accessible and is able to interpret simple block diagram models. Unfortunately, due to time constraints the demonstration adapter for converting the block model into a digital logic model was not completed. Additionally, further work is required to synchronise the UI between multiple control panels.

\begin{table}[ht!]
	\center
	\begin{tabular}{$p{65mm}^l^p{65mm}} %$
		\hline
		\rowstyle{\bfseries}
		Feature & Complete & Incomplete \\
		\hline
		Surface Capture & \gtick & \\
		Human interface (Control Panel) & \gtick & full UI synchronisation between multiple clients \\
		External control interface (RPC) & \gtick & \\
		Configurable (Keystore, via the above) & \gtick & More \gls{cv} settings should be user configurable \\
		Real-time Computer Vision for \newline Block Diagram recognition & \gtick & \\
		Block Diagram Model output \newline (MQTT Publish/Subscribe) & \gtick & \\
		\ldots & & \\
		Digital Logic Adapter & & \rcross \\
		Minimalist Digital Logic Simulator & & \rcross \\
		\hline
	\end{tabular}
	\caption{Feature completion for BlocSim prototype}
	\label{tab:completion}
\end{table} 

\clearpage

\section{Performance}

%TODO


\subsection{Speed \& Resource Usage}

\begin{description}
	\item[Network] - For a 1080p image cropped to approximately 50\% of its width and height, the jpeg frame size after base64 encoding is approximately 60KB. At a nominal frame rate of 5FPS, this gives 2.4mbps bandwidth for video, which is quite manageable on local networks. A blocsim server intended for remote access over the web should be constrained to a lower frame rate using the UI controls.
	\item[CPU] - This has not been extensively tested. On a MacBook Pro Retina with a 2.4GHz i5 processor, 8GB RAM, and integrated Intel graphics, the server ran at approximately 70\% CPU usage with one client attached (local). This is acceptable for personal use on a modern middle-range computer. 
	\item[Frame Rate] - At a full 1080p on the above system, BlocSim was able to maintain image processing at the speed-limited 5FPS.
\begin{description}


\subsection{Computer Vision Reliability}

\begin{figure}[ht!]
\centering
\includegraphics[width=125mm]{images/frame4glow.jpg}
\caption{BlocSim output image (4) - reflections on whiteboard can result in missed block components}
\label{im:frame4glow}
\end{figure}

%TODO

\section{Design Decisions Review}

%TODO

\section{Future Improvement}

%TODO



% =====================================================================

\chapter{Conclusion}

The creation of the BlocSim prototype according to speicifications laid out in Chapter \ref{ch:spec} was generally a success. Recognition of the whiteboard block diagrams runs in real time and performs well under controlled lighting conditions, however requires some initial manual tuning according to lighting conditions and camera. Control of the system and access to the block diagram model is convenient and user friendly through both \gls{gui} and external software, although the calibration controls should be expanded and synchronisation of controls between multiple web clients completed. The planned adapter for demonstrating conversion of generic block diagram models to digital logic gate models is yet to be implemented.
\\

Work on BlocSim so far has demonstrated that computer vision based primarily on the colour of whiteboard marker features is not as reliable as is desired. Even with a high resolution image, variations in actual marker colour, thin or lightly drawn features and strong reflections from background light sources can be significant, in some cases causing the algorithm to miss image features and build an incomplete block diagram model. This issue can be partially resolved in future work by encoding the block type in black-and-white frame marker fiducials, and pre-printing them onto non-reflective magnetic cards (see Section \ref{ch:back:borders}).
\\

Overall, the current prototype is a useful but small first step towards bridging the gap between block diagram sketches and simulation software. BlocSim will require extensive future work as an open source project before it can truly be a solution to this problem.

% =====================================================================

\begin{comment}

\end{comment}
