\mainmatter

% =====================================================================

\chapter{Introduction}

A block diagram is a model of a system in which components or functions are represented by blocks, and relationships or connections between them by connecting lines. Block diagram models of engineering systems are heavily used by designers and educators in many fields, including hardware design, electronics design, software engineering, control systems and others.
\\

Sketching block diagrams (or block diagram equivalent designs) on a static surface (such as paper or a whiteboard) is a common part of the design process for engineering students, educators and professionals. Often the system being designed is subsequently analysed and simulated using a computer aided tool, requiring the sketched design to be manually reconstructed on a computer. It should be possible to reduce or eliminate this delay between sketch and simulation and thus increase productivity for system designers using an automated, computer aided process.
\\

Today there are plenty of available systems capable of handwriting recognition for various engineering designs. However, these are generally specialised for taking input from a stylus or touchscreen, and/or recognition of symbols and designs for a specific engineering domain.
\\

\textit{BlocSim}\footnote{BlocSim - http://www.github.com/CPonty/BlocSim} is a proof-of-concept for a sketch-to-simulation solution capable of servicing a broad range of engineering domains. Its role is to visually capture a sketch of a block-diagram-like model and update external simulation tools with an adapted model in real-time. 
\\

An open-source prototype of BlocSim shall be built which achieves this functionality for simple whiteboard sketches. BlocSim will be built with extensibility in mind, as expanding its ability to interface with third-party software will require updates over time.


\begin{comment}

\end{comment}

% =====================================================================

\chapter{Background}



\section{Block Diagrams in Engineering}

\begin{figure}[ht!]
\includegraphics[width=70mm]{images/blockDiagram.png}
\hspace{0.5 cm}
\includegraphics[width=50mm]{images/circuitDiagram2.png} \\
\includegraphics[width=50mm]{images/controlSystem.png} 
\hspace{0.5 cm}
\includegraphics[width=50mm]{images/fbd.jpg} 
\centering
\caption{Block diagram (top left) \cite{wiringDiagrams}, circuit diagram (top right) \cite{engineeronadisk}, control system block diagram (bottom left) \cite{engineeronadisk}, function block diagram (bottom right) \cite{wikicommonsFbd}}
\label{im:blockExample}
\end{figure}

%\clearpage



\section{Block Diagram Sketch Capture}



\subsection{Sketch Design Recognition \& Simulation}



\subsection{}



\subsection{Frame Markers}
\label{ch:back:borders}



\section{Computer Vision Software \& Techniques}

\begin{figure}[ht!]
\centering
\includegraphics[width=25mm]{images/OpenCVlogo.png}
\caption{The OpenCV Logo \cite{opencv}}
\label{im:cvlogo}
\end{figure}


\section{Open Source Software}




% =====================================================================

\chapter{BlocSim Specification}
\ref{ch:spec}



\section{Solution Concept}

\begin{figure}[ht!]
\centering
\includegraphics[width=150mm]{images/BlocsimBD1.jpg}
\caption{BlocSim System Design \cite{blocsimPoster}}
\label{im:BlocsimBD1}
\end{figure}

\clearpage



\section{Block Diagram Capture}



\subsection{Fiducial Design}

\begin{figure}[ht!]
\centering
\includegraphics[width=100mm]{images/BlocsimBloc1.jpg}
\caption{BlocSim Block Diagram format (sketch)}
\label{im:BlocsimBD1}
\end{figure}

\begin{figure}[ht!]
\centering
\includegraphics[width=100mm]{images/BlocsimBloc2.jpg}
\caption{BlocSim Block Diagram format (model)}
\label{im:BlocsimBD2}
\end{figure}

\begin{figure}[ht!]
\centering
\includegraphics[width=100mm]{images/BlocsimBloc3.jpg}
\caption{BlocSim Block Diagram format (block types A,B,C,D mapped to Digital Logic gates)}
\label{im:BlocsimBD3}
\end{figure}

\clearpage



\subsection{Computer Vision Requirements}



\newpage

\section{Technology Selection}


\begin{table}[ht!]
	\center
	\begin{tabular}{$r^l} %$
		\hline
		\rowstyle{\bfseries}
		Component & Selected Technology \\
		\hline \\
		Primary Programming Language & \textbf{Python} \footnotemark[1] \\
		Computer Vision Package & \textbf{OpenCV} \footnotemark[2] \\
		Human Interface & Web control panel \\
		Web Framework & \textbf{Tornado} \footnotemark[3] \\
		Communication/Storage Format & \textbf{\gls{json}} \footnotemark[4] \\
		Remote Access & \textbf{JSON-RPC} \footnotemark[5] protocol via \textbf{TornadoRPC} \footnotemark[6] \\
		Client-Server Communication & \textbf{Websockets} \footnotemark[7], using \textbf{SockJS} \footnotemark[8] \& \textbf{SockJS-Tornado} \footnotemark[9] \\
		Block Diagram Publishing & \textbf{Mosquitto} \footnotemark[10] Publish/Subscribe Broker \\
		\\
		\hline
	\end{tabular}
	\caption{Selection of technologies for the BlocSim prototype}
	\label{tab:techSelect}
\end{table} 

%\footnotemark[11]
%\footnotemark[12]
%\footnotemark[13]

\footnotetext[1]{Python - \url{https://www.python.org/}}
\footnotetext[2]{OpenCV - \url{http://opencv.org/}} 
\footnotetext[3]{Tornado - \url{http://www.tornadoweb.org/en/stable/}}
\footnotetext[4]{JSON - \url{http://json.org/}}
\footnotetext[5]{JSON-RPC - \url{http://json-rpc.org/wiki/specification}}
\footnotetext[6]{TornadoRPC - \url{https://github.com/joshmarshall/tornadorpc}}
\footnotetext[7]{Websockets - \url{http://www.websocket.org/}}
\footnotetext[8]{SockJS - \url{https://github.com/sockjs/sockjs-client}}
\footnotetext[9]{SockJS-Tornado - \url{https://github.com/mrjoes/sockjs-tornado}}
\footnotetext[10]{Mosquitto - \url{http://mosquitto.org/}}
\footnotetext[11]{Flask - \url{http://flask.pocoo.org/}}
\footnotetext[12]{Socket.io - \url{http://socket.io/}}
\footnotetext[13]{Redis - \url{http://redis.io/}}


\clearpage


% =====================================================================

\chapter{Implementation}



\begin{figure}[ht!]
\centering
\includegraphics[width=50mm]{images/BlocsimLogo.png}
\caption{The BlocSim Logo \cite{blocsim}}
\label{im:blocsimlogo}
\end{figure}

\clearpage



\begin{figure}[ht!]
\centering
\includegraphics[width=125mm]{images/photo2.jpg}
\caption{Camera and whiteboard demonstration setup}
\label{im:hardware1}
\end{figure}

\begin{figure}[ht!]
\centering
\includegraphics[width=75mm]{images/photo3.jpg}
\includegraphics[width=85mm]{images/photo4.jpg}
\caption{Logitech C920HD 1080p Camera; close-up of block diagram on whiteboard}
\label{im:hardware2}
\end{figure}

\clearpage


\section{Features \& Usage}

\begin{figure}[ht!]
\centering
\includegraphics[width=150mm]{images/screenshot_cv1.png}
\caption{The BlocSim Web Control Panel \cite{blocsim}}
\label{im:screenshot_cv1}
\end{figure}

\begin{figure}[ht!]
\centering
\includegraphics[width=150mm]{images/screenshot_calib.png}
\caption{BlocSim Control Panel - calibration sliders \cite{blocsim}}
\label{im:screenshot_calib}
\end{figure}

\begin{figure}[ht!]
\centering
\includegraphics[width=150mm]{images/screenshot_pubsub.png}
\caption{BlocSim Control Panel - Block Diagram model \cite{blocsim}}
\label{im:screenshot_pubsub}
\end{figure}

\begin{figure}[ht!]
\centering
\includegraphics[width=150mm]{images/screenshot_rpc.png}
\caption{BlocSim Control Panel - Remote Procedure Call server status \cite{blocsim}}
\label{im:screenshot_rpc}
\end{figure}

\clearpage





\begin{figure}[ht!]
\centering
\includegraphics[width=125mm]{images/frame1.jpg}
\caption{BlocSim output image (1) - cropped frame \cite{blocsim}}
\label{im:frame1}
\end{figure}

\begin{figure}[ht!]
\centering
\includegraphics[width=125mm]{images/frame3.jpg}
\caption{BlocSim output image (3) - red \cite{blocsim}}
\label{im:frame3}
\end{figure}

\begin{figure}[ht!]
\centering
\includegraphics[width=125mm]{images/frame4.jpg}
\caption{BlocSim output image (4) - red block \cite{blocsim}}
\label{im:frame4}
\end{figure}

\begin{figure}[ht!]
\centering
\includegraphics[width=125mm]{images/frame8.jpg}
\caption{BlocSim output image (8) - block ID \cite{blocsim}}
\label{im:frame8}
\end{figure}

\begin{figure}[ht!]
\centering
\includegraphics[width=125mm]{images/frame12.jpg}
\caption{BlocSim output image (12) - connector lines (mask) \cite{blocsim}}
\label{im:frame12}
\end{figure}

\begin{figure}[ht!]
\centering
\includegraphics[width=125mm]{images/frame13.jpg}
\caption{BlocSim output image (13) - connection nodes \cite{blocsim}}
\label{im:frame13}
\end{figure}

\begin{figure}[ht!]
\centering
\includegraphics[width=125mm]{images/frame15.jpg}
\caption{BlocSim output image (15) - block diagram overlay \cite{blocsim}}
\label{im:frame15}
\end{figure}

\begin{figure}[ht!]
\centering
\includegraphics[width=125mm]{images/frame16.jpg}
\caption{BlocSim output image (16) - block model \cite{blocsim}}
\label{im:frame16}
\end{figure}

\clearpage



\newpage
\begin{figure}[ht!]
	\fontsize{8pt}{8pt}
	\singlespacing
	\begin{mdframed}
		\verbinput{images/example.json}
	\end{mdframed}
	\caption{Sample of JSON Block Diagram Model output (not all blocks, nodes shown)}
	\label{tab:json}
\end{figure}

\clearpage




% =====================================================================

\section{Technical Architecture}

\begin{figure}[ht!]
\centering
\includegraphics[width=150mm]{images/BlocsimBD3.jpg}
\caption{BlocSim's Architecture}
\label{im:BlocsimBD3}
\end{figure}

\begin{figure}[ht!]
\centering
\includegraphics[width=150mm]{images/BlocsimBD4.jpg}
\caption{Process flow for video stream}
\label{im:BlocsimBD4}
\end{figure}

%\newpage
\clearpage

The Python webserver's functionality is split into several threads, including:

\begin{description}
	\item[Main Thread] - Executes the Tornado webserver IO Loop. Includes handling websockets and Remote Procedure Calls
	\item[Publish Thread] - Executes the Mosquitto client IO Loop
	\item[Capture Thread] - Reads from the camera when signalled
	\item[Webcam Thread] - Handles webcam connection/disconnection and storing frames
	\item[Processing Thread] - Uses OpenCV to generate images and the block model
	\item[Timer Thread] - Limits the webcam capture frame rate
\end{description}

These threads communicate via a series of event signallers and shared resources, such as the keystore database and image processing output.



% =====================================================================

\section{Open-Source Delivery}

The BlocSim source, release build, issue tracking and documentation are publicly hosted online at GitHub \cite{blocsim}\cite{github}. Having a centralised, publicly visible location for these services and material provides both exposure for the project and valuable support infrastructure for any future work on the project (by its original author or others) which may take place. BlocSim is licensed under GPLv2 \cite{gplv2}, a permissive license which guarantees freedom to share and change the software, appropriate for encouraging and facilitating open source development.
\\

For more information, see the resources outlined in Appendix \ref{ch:appendix}.



% =====================================================================

\chapter{Project Evaluation}



\section{Completion}

\begin{table}[ht!]
	\center
	\begin{tabular}{$p{65mm}^l^p{65mm}} %$
		\hline
		\rowstyle{\bfseries}
		Feature & Complete & Incomplete \\
		\hline
		Surface Capture & \gtick & \\
		Human interface (Control Panel) & \gtick & full UI synchronisation between multiple clients \\
		External control interface (RPC) & \gtick & \\
		Configurable (Keystore, via the above) & \gtick & More \gls{cv} settings should be user configurable \\
		Real-time Computer Vision for \newline Block Diagram recognition & \gtick & \\
		Block Diagram Model output \newline (MQTT Publish/Subscribe) & \gtick & \\
		\ldots & & \\
		Digital Logic Adapter & & \rcross \\
		Minimalist Digital Logic Simulator & & \rcross \\
		\hline
	\end{tabular}
	\caption{Feature completion for BlocSim prototype}
	\label{tab:completion}
\end{table} 

\clearpage



\section{Performance}



\subsection{Speed \& Resource Usage}



%\subsubsection{Network}
\paragraph{Network}



%\subsubsection{CPU}
\paragraph{CPU}



%\subsubsection{Frame Rate}
\paragraph{Frame Rate}



\subsection{Computer Vision Reliability}

\begin{figure}[ht!]
\centering
\includegraphics[width=125mm]{images/frame4glow.jpg}
\caption{BlocSim output image (4) - reflections on whiteboard can result in missed block components}
\label{im:frame4glow}
\end{figure}


\section{Design Decisions Review}



\section{Future Improvement}



% =====================================================================

\chapter{Conclusion}

The creation of the BlocSim prototype according to speicifications laid out in Chapter \ref{ch:spec} was generally a success. Recognition of the whiteboard block diagrams runs in real time and performs well under controlled lighting conditions, however requires some initial manual tuning according to lighting conditions and camera. Control of the system and access to the block diagram model is convenient and user friendly through both \gls{gui} and external software, although the calibration controls should be expanded and synchronisation of controls between multiple web clients completed. The planned adapter for demonstrating conversion of generic block diagram models to digital logic gate models is yet to be implemented.
\\

Work on BlocSim so far has demonstrated that computer vision based primarily on the colour of whiteboard marker features is not as reliable as is desired. Even with a high resolution image, variations in actual marker colour, thin or lightly drawn features and strong reflections from background light sources can be significant, in some cases causing the algorithm to miss image features and build an incomplete block diagram model. This issue should be circumvented in future work by encoding the block identites in black-and-white frame marker fiducials, which are pre-printed onto non-reflective magnetic cards (see Section \ref{ch:back:borders}).
\\

Overall, the current prototype is a useful but small first step towards bridging the gap between block diagram sketches and simulation software. BlocSim will require extensive future work as an open source project before it can truly be a solution to this problem.

% =====================================================================

\begin{comment}

\end{comment}
